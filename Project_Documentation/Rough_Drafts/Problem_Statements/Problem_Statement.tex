\documentclass[onecolumn, draftclsnofoot,10pt, compsoc]{IEEEtran}
\usepackage{graphicx}
\usepackage{url}
\usepackage{setspace}

\usepackage{geometry}
\geometry{textheight=9.5in, textwidth=7in}

\def \DocType{Problem Statement}

\begin{document}
\begin{titlepage}
\pagenumbering{gobble}
    \begin{singlespace}
    		
		\hfill  
        \par\vspace{.2in}
        \centering
        \scshape{
            \LARGE CS Capstone \DocType \par
            {\large\today}\par
            \vspace{1in}
            \textbf{\huge Collaborative Threat Mitigation}
            \vspace{.4in}
            {\Large{ 
            	\\Oregon State University\\CS 461\\
            	\vspace{.4in}
                Prepared for:\\ Lonnie Mandigo\\
            	\vspace{.4in}
            	Prepared By:\\ Kyle Prouty\\Hayden Anderson\\Lucien Tamno 
                }
            }
            \par
            \vfill
        }
    
        \begin{abstract}
        \noindent
The world around us is filling with more and more interconnected devices that take up more and more bandwidth on our networks but currently these devices are not able to communicate or collaborate. This project aims to allow these devices to stop talking to the outside world and start talking to each other in order to self-organize to achieve an objective. Our challenge will be to create a network of loosely coupled wireless devices that are able to gather information about the world around them in order to collectively mitigate security threats that we introduce into the system. To solve this problem we will be building a framework that allows multiple devices to share processes in order to work together on a shared objective. By doing this, we can improve wireless devices in the future so that they can communicate with each other in order to collaborate on objectives without having any rigid logical configurations.
        \end{abstract}      
        
    \end{singlespace}
\end{titlepage}


\newpage
\pagenumbering{arabic}
\tableofcontents
\newpage

\section{Problem Definition}
\subsection{Overview}
The main problem that needs to be solved is to build framework that allows wirelessly connected devices to communicate with each other, share processes, and in the end collaborate on a shared objective.
\subsection{Devices}
Another problem that we will solve is to build multiple devices that will act as nodes on our network. To do this we will need to build devices that will be able to automatically connect to each other. This involves creating devices that when added to the network will be able to gather information about the services and state of the network and other nodes. These devices will be low powered and will not be able to receive configuration commands from any outside network or service. To build these devices we will be using various micro-controllers. Devices will need to be low powered, low bandwidth, and have flexible logical configurations.
\subsection{Services}
Every device node on the network will need to offer services. Devices will need to be able to communicate with other devices and gather information about what services they can offer. A problem to solve will be to allow these services to be vulnerable to security threats that we introduce into the network. Services will need to be able to be infected with a security threat which needs to disable temporarily or be permanently shut down the device. 
\subsection{Mitigation and Threats}
Once we have solved the above problems, then we will be ready to solve the problem of introducing threats into the system. Determining what a threat is and how it disrupts a service will be a problem to solve. We will need the ability to infect a service that is currently being offered by one of the devices on the network. If a service is threatened we will need to find a way to get the devices to share their resources so that other devices can take over the threatened services which mitigates that threat.  



\vspace{0.0 in}
\section{Proposed Solution}
\subsection{Self Organization}
The solution to our main problem will be to create a framework that allows wireless devices to communicate with each other and share their processes. This solution will entail developing this framework so that these devices can self-organize on an objective. Our team will be creating a network of devices using various wireless micro-controllers. The creation of these devices will include the ability of the connected devices to be able to gather information about other device devices. This includes the ability for them to check on the current state of the system and services that are available to be offered. Once devices can communicate and share their processes and all services offered, they will then be able to self-organize around a shared goal.
\subsection{Threat Introduction}
For us to solve this problem we will develop the ability to introduce threats onto the network. Threats will be manually introduced onto the network and able to infect devices. These threats will have the ability to shut down services. After a threat has shutdown a service, other devices will see that a device has been compromised. When a device has been determined to have been compromised, the other node will then self organize to mitigate the threats on the compromised node. This will be accomplished by knowing what services the other devices offer and then sharing those services so that an uninfected device can take over and offer the services that were infected. Introduced threats into the network will have different levels of threat. Device nodes will use information about the level of threat introduced when deciding how to mitigate that threat. Using these threat levels devices can make different decisions. They could decide to just disable services offered by that node, or if a high level threat has been introduced, they could be able to decide if an entire device and all of its services should be shut down.
\subsection{Building Devices}
To solve this problem we will build a network of loosely coupled wireless devices using various micro-controllers. Devices will be low powered, have low bandwidth usage, and have no rigid logical configurations. They will automatically connect themselves to our mesh network and start communicating with other devices. The devices will have the ability to share their processes and have processes shared with them. These devices will give us the ability to test how real world devices can utilize our framework to share processes in order to collaborate on an objective. 



\section{Performance Metrics}
\subsection{Prototype}
To show that threats can be mitigated our team will be creating a prototype of this network. This prototype will include the ability for wireless devices to communicate with each other, share resources, and collaborate on an objective. This prototype will include the ability for nodes to be added and subtracted from the system. It will include the ability for devices introduced to the system to be able to automatically connect and discover the current state and services offered by the other nodes already on the system. The prototype system will be able to have a threat introduced which shuts down services on individual nodes. 
\subsection{Display}
In order to show that our solution is working correctly we will develop a graphical interface which will show the system as it moves through the varying states of the threat mitigation. This display will show nodes connected, services offered, threats introduced, and how the devices are sharing processes. Furthermore, this display will show in real time how the threat is moving through the system, how the devices respond, then how the threat is mitigated, and finally show how the beginning and ending states compare.
\subsection{Testing}
The last key to judge the outcome of our project will be to create a test suite. This test suite will test for the optimum circumstances of the devices; making sure that they can join and leave the network correctly and seamlessly. Testing that devices can gather information from other devices and able to share processes. Another major section will be testing against different threat scenarios to ensure that our system is correctly mitigating security threats.  



\vspace{2 in}

\end{document}